
\section{Experimentos}
La obtención de unos algoritmos que lograron unos resultados satisfactorios ha sido una ardua labor. Los problemas con el sobreajuste de los modelos (ver \ref{section:sobreajuste} siempre ha estado presente, en gran parte, debido al poco número de ejemplares con los que se dispone para el entrenamiento y la estrecha relación de los índices del tensor. Además, también ha causado que no todos los algoritmos respondieran  como se esperaba, ni siquiera tras buscar intensamente una configuración apropiada. 

Los algoritmos más simples han proporcionado unos mejores resultados y apenas eran propicios al sobreajuste. En cambio, cuando se aumentó la complejidad de los algoritmos, los resultados ya no eran tan satisfactorios. Por ejemplo, se intentó incrementar el número de capas ocultas en la ``Artificial Neural Network'' pero todos los resultados empeoraron notablemente. Por este motivo, se han descartado un gran número de algoritmos que no lograban estimar  las clases que representan las disfunciones cognitivas.

Finalmente se han seleccionado cinco modelos que han dado una respuesta muy esperanzadora. Como se puede ver en la tabla resumen \ref{table:resultados}, tres de los cinco han superado un acierto del 70\% al introducir todas las matrices de conectividad en el modelo de predicción de  la variable clínica (rendimiento cognitivo). Cuando se han usado solamente las matrices relacionadas con el tensor de difusión (índices sensibles a la microestructura del tejido subyacente), los resultados han mejorado considerablemente, llegando a una predicción del 75\% e incluso superando el 80\% en algunos casos. Esto ocurre también cuando buscamos una mejor configuración, como se ven en las imágenes 
\ref{figure:logall}, 
\ref{figure:svmall}, 
\ref{figure:bayesall}, 
\ref{figure:forestall} y
\ref{figure:annall}, donde se muestran las configuraciones de los modelos que dan mejores resultados.  A pesar de contar con todas las matrices de conectividad para la predicción del rendimiento cognitivo, el mejor modelo de predicción incluye únicamente a las matrices \gls{dti}. Señalando que estos datos son quizás los más sensibles para la manifestación de la disfunción cognitiva en los pacientes con \gls{em}.

Estos resultados, a pesar de ser convincentes, no pueden darse como definitivos ya que la búsqueda de los parámetros no ha sido posible ejecutarla completamente con todas las combinaciones posibles. Dado el elevado número de posibles combinaciones de los hiperparámentros, es inviable ejecutar todas ellas con el sistema disponible. Por ello se ha seleccionado la configuración entre 150 combinaciones aleatorias. Por otra parte, se aprecia claramente la importancia de los índices \gls{dti} sobre el resto y cómo éstas pueden son una fuente de datos válida para la estimación de las disfunciones cognitivas.

\section{Proyecto de investigación}
La realización de este trabajo ha sido un gran reto personal. Es la primera vez que me enfrento a un trabajo de investigación y, a la vez, a un ejercicio relacionado con el aprendizaje automático en el ámbito clínico. Todo esto conjuntamente con las lecciones aprendidas durante la realización del trabajo y los resultados obtenidos me han proporcionado experiencia muy positiva y enriquecedora.

La primera dificultad encontrada durante este trabajo fue la ``compresión del negocio''. Los datos usados en este trabajo tienen tras de sí múltiples investigaciones, teorías y técnicas que ha sido necesarias para poder comprender y trabajar con los datos. El conocimiento adquirido durante esta fase inicial ha sido de gran utilidad a lo largo de la ejecución del proyecto.

Uno de los mayores retos durante el proceso de investigación ha sido la búsqueda de algoritmos capaces de dar respuesta a las objetivos planteados dado el gran número de algoritmos disponibles y todas las opciones que el aprendizaje automático provee. Además, siempre hay que considerar los requerimientos y el tiempo de cómputo que los algoritmos necesitan. A pesar de las implementaciones óptimas ofrecidas por los  ``frameworks'', el hardware usado se ha visto a menudo incapaz de soportar la carga de trabajo en un tiempo razonable. Para lidiar en parte con este problema se optó por la búsqueda aleatoria de la configuración de los algoritmos seleccionados. 

En definitiva, el objetivo inicial planteado para la superación del trabajo final de Máster de ciencia de datos ha sido cumplido. Los resultados obtenidos demuestran  cómo los algoritmos de aprendizaje automático pueden ayudar a la predicción del rendimiento cognitivo del paciente con \gls{em} a través de la cuantificación del volumen lesional y la microestructura de la red cerebral.
