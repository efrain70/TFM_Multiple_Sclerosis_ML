Este trabajo de investigación se puede considerar como un primer paso para una investigación más profunda. La obtención de una cohorte mayor y la introducción de nuevos datos relacionados con la disfunción cognitiva, como por ejemplo medidas de atrofia, podrían mejorar la eficiencia de las técnicas de aprendizaje automático. 

En primer lugar, sería muy interesante comprobar los resultados obtenidos con una fuente de datos más amplia y poder investigar la configuración óptima de los algoritmos con más profundidad. 

Por falta de tiempo se han quedado sin ejecutar algunas propuestas para la reducción de la dimensionalidad, por ejemplo \gls{csp}, y también técnicas que combinan clasificadores sencillos para obtener un más complejo (bagging y stacking). Experimentos con estas técnicas también podrían producir resultados interesantes.
