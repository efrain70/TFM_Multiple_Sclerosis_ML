Las técnicas de aprendizaje automático ha aportado grandes avances en estudios relacionados con la estructura cerebral y la neuroimagen. Por ejemplo, \cite{Zhang2016ComparisonMachine} el uso de la herramienta ``entropía de ondas estacionarias'' (SWE) permitió extraer aquellos  atributos más característicos de las imágenes cerebrales. Posteriormente la aplicación del \gls{knn}, consiguió una clasificación del 97.94\% de exactitud diferenciando sujetos sanos y pacientes con \gls{em}. Otros estudios, basándose en  técnicas de aprendizaje automático, consiguieron localizar \glspl{roi} partiendo de imágenes de \gls{rm} \cite{Desikan2006AnInterest}.

Si bien la aplicación de \glspl{ann} para el análisis de imágenes médicas está muy extendido, su uso en aplicaciones neurológicas está aún en desarrollo \cite{Ciresan2013MitosisNetworks} \cite{Ciresan2012DeepImages}. A través de algoritmos más complejos de aprendizaje automático, ``deep learning'' o aprendizaje profundo, se han obtenido resultados muy prometedores en este campo. Por ejemplo, usando una \gls{cnn} con estructura LeNet-5, se ha logrado clasificar  satisfactoriamente enfermos de Alzheimer con una exactitud de 96.85\% a partir de imágenes por \gls{frm}. Es más, con este método también se consiguió predecir los estados de la enfermedad para los diferentes grupos de edades \cite{Sarraf2016ClassificationNetworks}. También otros estudios relacionados con \gls{em} han demostrado la eficacia de este tipo de algoritmos. Por ejemplo, el uso conjunto de \gls{rbm} y \gls{dbn}  ha demostrado ser una herramienta eficaz para la creación de características nuevas tan eficaces como las obtenidas directamente con los datos \cite{Yoo2014DeepSegmentation}.


Teniendo esto presente, se han creado herramientas específicas para las aplicaciones neurológicas. Por ejemplo, ann4brains \cite{Kawahara2017BrainNetCNN:Neurodevelopment} proporciona redes neuronales concebidas especialmente para el uso de matrices de conectividad. 

Destacar también que la mayoría de los estudios que aplican el aprendizaje automático a datos del conectoma se han centrado en predecir los resultados a través de la clasificación o regresión. Además, el uso del aprendizaje automático ha permitido identificar subredes y regiones de interés a través de un análisis no supervisado.

También está muy extendido en la actualidad el uso de algoritmos relacionados con \glspl{svm} \cite{HamarnehsGroup} ya que ofrecen una mejor precisión de predicción y es menos sensible al ruido. Por ejemplo, \cite{Craddock2009DiseaseConnectivity} se ha conseguido alcanzar una exactitud del 95\% para distinguir entre una cohorte de sujetos sanos y otra de pacientes afectados por depresión a través de imágenes de \gls{frm}.

En el ámbito de la neurociencia computacional los datos utilizados como entrada suelen tener una relación entre el número de ejemplares y la dimensión de estos bastante desfavorable. Para hacer factible la tarea del aprendizaje en muchos de estos estudios, la dimensionalidad de los datos debe reducirse significativamente para poder extraer las características más relevantes. Los algoritmos no supervisados más usados para este propósito son \gls{pca} y \gls{ica}, entre los supervisados destaca \gls{csp} \cite{Lemm2011IntroductionImaging}.
