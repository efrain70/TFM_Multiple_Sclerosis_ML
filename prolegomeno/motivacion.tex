Durante todo el proceso de aprendizaje del Máster en Ciencias de Datos me han atraído principalmente las materias relacionadas con el aprendizaje automático y cómo estos algoritmos pueden ayudar a transformar los numerosos datos en conocimiento para nuestra sociedad. 

Concretamente, la investigación médica se mueve en un ámbito global donde se generan millones de datos diariamente sobre pacientes afectados por alguna dolencia o enfermedad. Este escenario es propicio para la aplicación de técnicas de aprendizaje automático ya que su empleo  pueden ayudar a analizar multitud de datos conjuntamente y ser capaces de detectar patrones imperceptibles para el ojo  humano.

Gracias a la colaboración en la investigación junto con el grupo de Imagen Avanzada en enfermedades Neuroinmunológicas (ImaginEM) sobre esclerosis múltiple tengo la posibilidad de poder integrar mi experiencia en las tecnologías de programación y en la ingeniería del software con los nuevos conocimientos adquiridos en el Máster en un marco hospitalario.
