
La \gls{em} es una enfermedad neurodegenerativa crónica, inflamatoria y desmielinizante del \gls{snc}, de carácter autoinmune y considerada como una de las causas más importantes de discapacidad física y cognitiva en adultos jóvenes \cite{Rocca2015ClinicalSclerosis}. Esta enfermedad se caracteriza principalmente por la presencia de placas desmielinizadas focales, y por una afectación microestructural difusa más allá de las lesiones en la gls{sban} y la \gls{sgan}. Ambos componentes son los responsables de la atrofia cerebral y, en cierta medida, están asociados con la discapacidad cognitiva \cite{Kutzelnigg2014PathologyDiseases}. Este deterioro cognitivo está presente en el 40-70\% de los pacientes, incluso durante etapas tempranas de la enfermedad. Los déficits cognitivos más comunes son los relacionados con la atención, las funciones ejecutivas, la velocidad de procesamiento de la información y la memoria episódica \cite{Chiaravalloti2008CognitiveSclerosisb}. 

La \gls{rm} convencional ha demostrado ser una  técnica útil para el diagnóstico y monitorización de la enfermedad de \gls{em}.  La presentación de lesiones características de la enfermedad se muestran hipointensas en secuencias potenciadas en T1 y con un aumento de señal en secuencias potenciadas en T2. Sin embargo, la presencia de lesiones y las manifestaciones clínicas de los pacientes son modestas \cite{Barkhof2002TheRevisited}. Probablemente, a causa de la baja especificidad en la patología subyacente y a la baja sensibilidad del daño del tejido de apariencia normal en este tipo de secuencias.  La existencia de nuevas modalidades de imagen por RM, denominadas \gls{rm} avanzada o no convencional, ha aportado información más sensible más allá de las lesiones focales, permitiendo estudiar el tejido de apariencia normal, siendo una de las modalidades más populares actualmente es la \gls{drm}. Esta técnica de \gls{rm} se basa en el estudio del movimiento de las moléculas de agua y permite caracterizar las trayectorias de \gls{sb} de forma no invasiva, dado que en la \gls{sb} este movimiento de moléculas de agua predomina en la dirección paralela al axón y se encuentra limitado en su dirección perpendicular \cite{Basser2000InDatab}.  Por tanto, a partir de la \gls{drm} podemos obtener datos cuantitativos y patológicamente más específicos capaces de detectar cambios en la integridad microestructural en \gls{sban} y \gls{sgan}. A partir de la \gls{drm} podemos obtener unas medidas que se conocen como \gls{dti}. Estas medidas cuantifican la direccionalidad y magnitud del movimiento de las moléculas de agua en el espacio tridimensional. Dado que existe una fuerte direccionalidad debido a la presencia de mielina y axones en la \gls{sb}, el movimiento es anisotrópico y puede representarse como una elipsoide, donde el semieje principal indica la máxima difusividad. La elipsoide se puede parametrizar por un conjunto de vectores y valores propios que nos proporcionan unos índices sensibles a la integridad del tejido. El más común es la \gls{fa}. La \gls{fa} es una variable numérica cuyos valores oscilan entre 0 (máxima isotropía) y 1 (máxima anisotropía). La \gls{fa} es mayor en la \gls{sb} que en la \gls{sg}, debido a que la movilidad del agua está altamente influenciada por la organización de las fibras nerviosas. Por este motivo, el valor de \gls{fa} es comúnmente utilizado en los estudios como un marcador de la integridad estructural, ya que la pérdida de barreras reduce el grado de anisotropía (menor \gls{fa}). Hay otros valores del tensor que se pueden cuantificar como: la \gls{md}, la \gls{ad} y la \gls{rd}. Mientras que la \gls{fa} y la \gls{md} se han asociado a diversos cambios patológicos en el tejido, los cambios de \gls{ad} y \gls{rd} se han asociado con daño axonal y desmielinización principalmente en estudios con modelos animales \cite{Song2005DemyelinationBrain}.

Mediante el tensor de difusión es posible generar una representación de las fibras de la \gls{sb} o tractografía. La tractografía utiliza la dirección de máxima difusividad entre vóxeles cercanos para trazar las diferentes conexiones que componen la red cerebral. Sin embargo, esta aproximación es muy simplista dado que el modelo por \gls{dti} no es capaz de descomponer las diferentes fibras contenidas en un solo vóxel. La estructura local de la SB presenta regiones de cruce, dobleces o dispersión de fibras en más del 90\% de los vóxeles \cite{Jeurissen2013InvestigatingImaging}, El uso de modelos avanzados de tractografía proporciona una representación de mayor resolución angular de los diferentes máximos locales contenidos dentro del vóxel y reemplaza la representación de la elipsoide por otra estimación más compleja \cite{Tuch2002HighHeterogeneity}. Gracias a esto se puede descomponer las fibras de \gls{sb} en diferentes direcciones en una región de cruce de fibras y reconstruir aquellos tractos que presentan una gran curvatura y una baja \gls{fa} \cite{Martinez-Heras2015ImprovedRadiation}.  

La realización de la modelos avanzados de tractografía permite generar fibras de \gls{sb} biológicamente más precisas respecto a la anatomía subyacente y trazar las trayectorias que componen la red cerebral \cite{Rubinov2010ComplexInterpretations}. Una vez generada la  tractografía es posible cuantificar la reconstrucción con diferentes índices como pueden ser el número de fibras, el volumen lesional de las conexiones o las medidas del tensor. Esto puede facilitar el uso de medidas sensibles para diferenciar entre un cerebro sano y otro con presencia de una patología. Además de la posibilidad de detectar qué conexiones del cerebro (subsistemas) guardan algún tipo de relación con variables clínicas de interés \cite{Llufriu2017StructuralSclerosis}. 

A través de la cuantificación de este tipo de medidas sensibles a la microestructura tisular se pueden relacionar cambios patológicos de la \gls{sban} con el deterioro cognitivo en pacientes con \gls{em} \cite{Gabilondo2014Trans-synapticSclerosis} \cite{Llufriu2017StructuralSclerosis}. Otra herramienta útil para la detección de anomalías de la red cerebral es la teoría de grafos \cite{Bullmore2009ComplexSystems}. Esta metodología permite caracterizar varios aspectos de la estructura de la red, designando las distintas regiones de interés (\gls{sg}) como nodos de un grafo y las conexiones entre estos nodos como aristas (\gls{sb}). Este planteamiento ha facilitado la comprensión de la arquitectura de la red cerebral. Basándose en estos métodos, estudios recientes han demostrado que el cerebro no puede ser considerado simplemente como una gran red interconectada, sino más bien una colección jerárquica de redes de ámbito local que cooperan paralelamente y son capaces de optimizar información por medio de diferentes estructuras modulares \cite{Sporns2011NetworksBrain}. 

El uso de herramientas de análisis de la conectividad estructural en los pacientes con \gls{em} ha demostrado que la red cerebral de estos pacientes tienen una menor capacidad para intercambiar y procesar información eficazmente. Además, la existencia de mecanismos de desconexión ha sido descrita como una causa importante en la manifestación de la alteración cognitiva en la \gls{em} \cite{Shu2011DiffusionSclerosis} \cite{Llufriu2017StructuralSclerosis} \cite{Bozzali2013AnatomicalSclerosis} \cite{Louapre2014BrainStudy} \cite{Dineen2009DisconnectionSclerosis}. Sin embargo, a pesar de estos importantes hallazgos, hay que tener en cuenta otros factores que pueden relacionarse con el rendimiento cognitivo como el volumen lesional y el daño tisular local de la \gls{sb} \cite{Stellmann2017ReducedMS} \cite{Ouellette2018LesionSclerosis.}. 

Como herramienta de análisis de este trabajo se introduce el aprendizaje automático para el estudio de la conectividad estructural de los pacientes con \gls{em}. El aprendizaje automático forma parte de las ciencias computacionales caracterizándose por la búsqueda de modelos capaces de generalizar comportamientos implícitos en los datos haciendo que "aprendan" automáticamente a partir de la información proporcionada. La tarea de clasificación es una de las más importantes entre todo el conjunto de algoritmos que se engloban en las técnicas de aprendizaje automático. Utilizando algoritmos supervisados, se caracteriza por la búsqueda de un modelo capaz de asignar un conjunto de clases, o etiquetas, basándose en la información contenida en los datos. Estas técnicas de aprendizaje automático han ayudado a numerosos estudios a obtener resultados muy positivos, como por ejemplo, a la hora de diferenciar entre pacientes con \gls{em} y sujetos sanos basándose en la información contenida en las imágenes de \gls{rm} \cite{Zhang2016ComparisonMachine}.
