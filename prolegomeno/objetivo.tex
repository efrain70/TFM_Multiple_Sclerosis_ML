El propósito de este estudio es intentar predecir el rendimiento cognitivo del paciente con \gls{em} a través de la cuantificación del volumen lesional y la microestructura de la red cerebral empleando herramientas de aprendizaje automático. 

Partiendo de matrices de conectividad estructural, calculadas mediante la estimación del volumen de lesión y de los índices del tensor en cada una de conexiones cerebrales, se estudiará cómo difieren los pacientes con un mayor o menor rendimiento cognitivo. Esto será útil para predecir cuándo la lesión cerebral implican disfunción cognitiva.