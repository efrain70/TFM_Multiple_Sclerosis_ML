\pagenumbering{roman} 
\setcounter{page}{1} 
\pagestyle{plain}

%\begin{abstract}
\chapter*{Resumen}
\addcontentsline{toc}{chapter}{Resumen}

%\singlespacing
\onehalfspacing

Los grafos son un formato de representaci�n complejo y flexible, que permite representar de una forma natural una gran diversidad de realidades. Algunos ejemplos de estos datos son: redes sociales, redes de comunicaciones, estructuras biol�gicas, etc.

En paralelo a la explotaci�n de este tipo de datos, aparecen los problemas de seguridad asociados a su difusi�n. Cuando se difunde un grafo se est�n difundiendo datos de los individuos que aparecen en �l, y algunos de ellos pueden ser datos sensibles o privados. Es necesario detectar y proteger las identidades de los individuos que aparecen en los grafos antes de proceder a su difusi�n.

En este trabajo se realiza una breve revisi�n del estado del arte en m�todos de anonimizaci�n de grafos. Para poder ver la problem�tica en toda su dimensi�n, tambi�n se revisan conceptos relacionados como las medidas de calidad o los m�todos de re-identificaci�n y conocimiento del adversario. Tambi�n se realiza una breve revisi�n sobre algunos m�todos de miner�a de datos aplicada a grafos (\emph{graph mining}).

A continuaci�n se escogen dos m�todos de anonimizaci�n y se analiza su comportamiento ante distintos conjuntos de datos reales. Se eval�a el grado de perturbaci�n introducido a partir de las propiedades estructurales y el grado de afectaci�n que pueda tener en el resultado de los procesos de \emph{graph mining} aplicados sobre los datos. Por otro lado, tambi�n se eval�a el nivel de seguridad de los datos anonimizados.

A partir de las deficiencias observadas en los dos m�todos de anonimizaci�n, se implementa un m�todo basado en anteriores estudios de Liu y Terzi. El nuevo m�todo es analizado con los mismos conjuntos de datos y demuestra superar algunas de las deficiencias detectadas en los m�todos anteriores.

\vspace*{1cm}

\textbf{Palabras clave:} privacidad, anonimizaci�n, grafos, miner�a de datos, \emph{graph mining}.
