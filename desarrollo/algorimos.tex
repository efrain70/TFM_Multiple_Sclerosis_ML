El aprendizaje automático es una rama de la inteligencia artificial cuyo objetivo es proporcionar técnicas para hacer que los sistemas “aprendan”. Este aprendizaje se basa en algoritmos que a partir de un conjunto de datos son capaces de crear un modelo capaz de generalizar comportamientos y reconocer patrones. Estos modelos se caracterizan por su finalidad y se clasifican en dos grupos diferenciados: Algoritmos supervisados y algoritmos no supervisados

\section{Tipos de tareas}
En función de la tarea que queramos resolver con el aprendizaje automático podemos diferenciar en tres grupos: clasificación, regresión y agrupamiento. Todas ellas siguen un paradigma inductivo donde, a partir de los datos y un modelo se puede obtener un nuevo conocimiento que puede ser aplicado posteriormente a nuevos datos. A su vez, éstos se diferencia en dos grandes grupos, algoritmos supervisados y no supervisados. 

\subsection{Clasificación}
La tarea de clasificación se centra en asignar un conjunto de clases a instancias de un dominio compuesto por atributos discretos o continuos. Estas clases, o etiquetas, tiene que ser conocidas para un subconjunto para la construcción del modelo.

\subsection{Regresión}
La tarea de regresión la podemos describir como una clasificación con clases continuas. Es decir, asigna un valor numérico a instancias de un dominio compuesto por atributos discretos o continuos. Este modelo está definido por una función, generalmente desconocida, que establece un valor numérico para los nuevos datos.

\subsection{Agrupamiento}
El agrupamiento, al igual que la clasificación y la regresión, es una tarea inductiva. En cambio, se considera un algoritmo no supervisado ya que no se dispone de un conjunto de clases para predecir. El resultado del agrupamiento es un conjunto de clases y la asignación de cada elemento del conjunto de datos a una de estas clases basándose en la similitud entre las instancias. El modelo que se obtiene como resultado del agrupamiento también asigna a cada nueva instancia una clase de las anteriormente obtenidas.

\section{Algoritmos supervisados y no supervisados}

\subsection{Algoritmos supervisados}
Estos algoritmos requieren un conjunto de datos etiquetados. Si las etiquetas de los datos son categóricos se denominan algoritmos de clasificación. En cambio, si son etiquetas numéricas corresponden a algoritmos de regresión. Por lo tanto, son usados tanto en tareas de clasificación como de regresión.

\subsection{Algoritmos no supervisados}
Estos algoritmos no necesitan que los datos dispongan de ninguna etiqueta o clasificación previa. Se centran en el agrupamiento o segmentación con el fin de encontrar características  similares. Podemos diferenciar entre  dos grupos; los métodos jerárquicos donde se obtiene una organización con varios niveles de agrupación, y los métodos particionales o no jerárquicos.
