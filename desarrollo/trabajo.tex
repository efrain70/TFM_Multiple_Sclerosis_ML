Los métodos del aprendizaje automático pueden ayudar a las deficiencias de la interpretación humana en relación con la complejidad del cerebro [1]. Más concretamente, usando algoritmos supervisados de clasificación predecir la disfunción cognitiva en pacientes de \gls{em}

Teniendo siempre presente esta meta, se ha llevado a cabo un estudio preliminar de los datos proporcionados. Posteriormente se implementó un sistema para la limpieza y preparación de los datos para la aplicación a los algoritmos de aprendizaje automático. Con especial énfasis en descartar modelos que sufrieran sobreajuste y usando de métodos de validación cruzada y remuestreo se han validado los resultados obtenidos siguiendo métricas de exactitud o ``accuracy''.