Este trabajo tiene como objetivo la predicción del rendimiento cognitivo del paciente con esclerosis múltiple (EM) a través de la cuantificación del volumen lesional y la microestructura de la red cerebral empleando herramientas de aprendizaje automático. Esta enfermedad neurodegenerativa es una de las causas más importantes de discapacidad física y cognitiva en adultos jóvenes.

Para esta investigación se ha dispuesto de una muestra de 182 sujetos, donde 140 padecen de EM y 42 son controles sanos. De cada uno de ellos se dispone de cuatro medidas de tensor de difusión (DTI) (Anisotropía fraccional, Difusividad media, Difusividad axial y Difusividad radial), número de fibras y volumen lesional. Toda esta información proveniente del análisis de la conectividad estructural  es presentada mediante matrices simétricas.

Tras realizar las tareas de preprocesamiento y limpieza de toda esta información, con el software NeuLoadData, se han estimado los mejores parámetros de configuración para los algoritmos ``Logistic Regression'', ``Support Vector Machine (SVM)'', ``Gaussian Naive Bayes, Random Forest Classifier'' y ``Artificial Neural Network (ANN)''. Usando únicamente las medidas de tensor de difusión todos los modelos obtenidos han sido capaces de predecir exitosamente más del 75\%. Por lo tanto, el enfoque propuesto de aprendizaje automático para la predicción el rendimiento cognitivo en pacientes con EM ha demostrado su utilidad e interés como herramienta para analizar un gran conjunto de datos satisfactoriamente en el campo sanitario.

\vspace*{1cm}

\textbf{Palabras clave:} aprendizaje automático, conectividad estructural, esclerosis múltiple, rendimiento cognitivo