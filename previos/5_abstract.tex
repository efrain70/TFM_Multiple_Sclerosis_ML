\vspace*{1cm}
The present study aims to predict the cognitive performance of patients with multiple sclerosis (MS) through quantification of lesional volume and the microstructure of the brain network by means of machine learning techniques. This neurodegenerative disease is one of the main causes of both physical and cognitive disability in young adults.

For this research, we have a sample of 140 patients with Multiple Sclerosis and 42 healthy controls. For each participant, information relating to four measures of diffusion tensor (DTI) (fractional anisotropy, medium diffusivity, axial diffusivity and radial diffusivity), number of fibers and lesional volume has been recorded. All this information coming from the analysis of structural connectivity is presented by symmetric matrices.

After carrying out preprocessing tasks on all of this information using the NeuLoadData software, the best configuration parameters have been estimated for the algorithms Logistic Regression, Support Vector Machine (SVM), Gaussian Naive Bayes, Random Forest Classifier and Artificial Neural Network (ANN). Making use of only diffusion tensor measurements, all the models have had a successful prediction rate of over 75\%. Therefore, the proposed approach of machine learning for prediction cognitive performance in patients with MS has demonstrated satisfactorily its usefulness and interest as a tool to analyze a large set of data in the health field.

\textbf{Keywords:} manchine learning, structural connectivity, multiple sclerosis, cognitive performance